\documentclass[10pt, letterpaper, sans]{moderncv}

\moderncvstyle{banking}  % style options are 'casual' (default), 'classic', 'oldstyle' and 'banking'
\moderncvcolor{blue}

% Ref for language-related: https://ptmartins.info/2021/05/15/cv.html
\usepackage[english,french]{babel}

\usepackage{siunitx}
\sisetup{detect-all=true}

\newcommand{\langen}[1]{\ifen\selectlanguage{english}#1\fi}
\newcommand{\langfr}[1]{\iffr\selectlanguage{french}#1\fi}

\newif\ifen
\newif\iffr

\entrue
% \frtrue

\usepackage{xcolor}
	\definecolor{udemblue}{HTML}{0047B6}
	\definecolor{cvblue}{HTML}{3873b2}

\AtEndPreamble{
  \hypersetup{
    colorlinks=true,
    linkcolor=cvblue,
    filecolor=cvblue,
    urlcolor=cvblue,
    citecolor=cvblue
  }
}
\urlstyle{same}


% Setup fonts
\usepackage[scaled]{helvet}
\renewcommand\familydefault{\sfdefault} 
\usepackage[T1]{fontenc}
\usepackage[utf8]{inputenc}
\renewcommand\familydefault{\sfdefault} 

% Wider page margins
\usepackage[scale=0.76]{geometry}

\usepackage{xspace}

\newcommand{\irex}{\langen{Trottier Institute for Research on Exoplanets (IREx)}\langfr{Institut Trottier de recherche sur les exoplanètes (IREx)}\xspace}

\providecommand{\money}[1]{\SI{#1}[\langen{\$}]{\langfr{\$}}}

% personal data
\name{Thomas}{Vandal}
%\title{Resumé title}                               % optional, remove / comment the line if not wanted
\address{  }
\email{thomas.vandal@umontreal.ca}                               % optional, remove / comment the line if not wanted
\social[github]{vandalt}
% \social[gitlab]{vandalt}
%\homepage{https://github.com/vandalt}                         % optional, remove / comment the line if not wanted
%\extrainfo{}
%\quote{}

% Legend for TODOs:
% BUG: Information is out of date

% TODO: Ponctuation uniforme dans les 3e ligne (point ou pas point)
% TODO: Save space with more descriptive titles and less "third line" info? eg.. tutorat -> tutorat en physique, math, lit?
\begin{document}

\langen{
  \sisetup{
    group-separator={,},
    group-minimum-digits=5
  }
}
\langfr{
  \sisetup{
    group-separator={ },
    group-minimum-digits=5
  }
}


\makecvtitle

\section{\langen{Education}\langfr{\'Education}}
\cventry{2020-...}{\langen{Astronomy and astrophysics}\langfr{Astronomie et astrophysique}}{\langen{PhD in Physics}\langfr{Doctorat en Physique}}{Montréal}{Université de Montréal}{\langen{Advisor: Prof. René Doyon, fast track from M.Sc. in May 2021}\langfr{Superviseur: Prof. René Doyon, passage accéléré de la maîtrise en mai 2021}}
\cventry{2017-2020}{\langen{Major Physics}\langfr{Majeure en physique}}{\langen{Bachelor of Science}\langfr{Baccalauréat en sciences} (B.Sc.)}{Montréal}{\langen{McGill University}\langfr{Université McGill}}{\langen{Final project: \textit{Modelling Stellar Activity with Gaussian Processes:
      Application to the $\beta$ Pictoris System}}\langfr{Projet final: \guillemotleft{} Modelling Stellar Activity with Gaussian Processes:
    Application to the $\beta$ Pictoris System \guillemotright{}} \\ \langfr{Superviseur:}\langen{Advisor:} Prof. René Doyon}  % arguments 3 to 6 can be left empty
\cventry{2015-2017}{\langen{Natural Sciences}\langfr{Sciences de la nature}}{\langen{Diploma of Collegial Studies}\langfr{Diplôme d'études collégiales}}{Shawinigan}{Collège Shawinigan}{}  % arguments 3 to 6 can be left empty

\section{\langen{Research Experience}\langfr{Expérience de recherche}}
% TODO: Review field description
\cventry{2020-...}{\irex{}}{\langen{PhD Student}\langfr{Étudiant au doctorat}}{Montréal}{Université de Montréal}{\langen{High-resolution spectroscopy, precise radial velocity, \\ aperture masking and kernel phase interferometry, orbit modelling}\langfr{Spectroscopie à haute résolution, vitesses radiales, modélisation d'orbites,\\ interférométrie à masque non redondant et par noyaux de phase}}
\cventry{2020-...}{Université de Montréal}{\langen{Research Assistant}\langfr{Assistant de recherche}}{Montréal}{}{\langen{Contribution to the APERO data reduction software for high resolution spectroscopy and precision radial velocity}\langfr{Contribution au logiciel de réduction de données \textit{APERO} pour la spectroscopie à haute résolution}}
\cventry{2018-2019}{\irex}{\langen{Undergraduate researcher}\langfr{Étudiant-chercheur au baccalauréat}}{Montréal}{}{\langfr{Modélisation de l'activité stellaire de $\beta$ Pictoris via des processus gaussiens}\langen{Disentangling Stellar Activity in $\beta$ Pictoris with Gaussian Processes} \\ \langen{Advisors}\langfr{Superviseurs}: Prof. René Doyon, Dr. Julien Rameau \langfr{et}\langen{and} Dr. Lauren Weiss
}
\cventry{2017}{AEPONYX}{\langen{Summer intern in photonics}\langfr{Stagiaire d'été en photonique}}{Trois-Rivières}{}{\langen{Computer-aided design, finite element analysis}\langfr{Dessin assisté par ordinateur, analyse d'éléments finis}}

\section{\langen{Publications and Presentations}\langfr{Publications et présentations}}
\subsection{\langen{Refereed Publications}\langfr{Articles dans une revue avec comité de lecture}}
\cvitem{}{\textbf{Vandal, T.}, Rameau, J., Doyon, R. \textit{Dynamical Mass Estimates of the {\ensuremath{\beta}} Pictoris Planetary System through Gaussian Process Stellar Activity Modeling}. 2020, \href{https://doi.org/10.3847/1538-3881/abba30}{AJ, 160, 243}.}
\cvitem{}{Parc, L. et al. (\langen{including}\langfr{incluant} \textbf{Vandal, T.}) \textit{NIRPS and TESS reveal a peculiar system around the M dwarf TOI-756: A transiting sub-Neptune and a cold eccentric giant}. 2025, \href{https://doi.org/10.1051/0004-6361/202555684}{A\&A, 702, A138}.}
\cvitem{}{Bazinet, L. et al. (\langen{including}\langfr{incluant} \textbf{Vandal, T.}) \textit{Quantifying thermal water dissociation in the dayside photosphere of WASP-121 b using NIRPS}. 2025, \href{https://doi.org/10.1051/0004-6361/202553724}{A\&A, 701, A276}.}
\cvitem{}{Gomes da Silva, J. et al. (\langen{including}\langfr{incluant} \textbf{Vandal, T.}) \textit{Blind search for activity-sensitive lines in the near-infrared using HARPS and NIRPS observations of Proxima and Gl 581}. 2025, \href{https://doi.org/10.1051/0004-6361/202555013}{A\&A, 700, A177}.}
\cvitem{}{Vaulato, V. et al. (\langen{including}\langfr{incluant} \textbf{Vandal, T.}) \textit{Hydride ion continuum hides absorption signatures in the NIRPS near-infrared transmission spectrum of the ultra-hot gas giant WASP-189b}. 2025, \href{https://doi.org/10.1051/0004-6361/202452972}{A\&A, 700, A9}.}
\cvitem{}{Bouchy, F. et al. (\langen{including}\langfr{incluant} \textbf{Vandal, T.}) \textit{NIRPS joining HARPS at ESO 3.6 m: On-sky performance and science objectives}. 2025, \href{https://doi.org/10.1051/0004-6361/202453341}{A\&A, 700, A10}.}
\cvitem{}{Allart, R. et al. (\langen{including}\langfr{incluant} \textbf{Vandal, T.}) \textit{NIRPS detection of delayed atmospheric escape from the warm and misaligned Saturn-mass exoplanet WASP-69 b}. 2025, \href{https://doi.org/10.1051/0004-6361/202452525}{A\&A, 700, A7}.}
\cvitem{}{Suárez Mascareño, A. et al. (\langen{including}\langfr{incluant} \textbf{Vandal, T.}) \textit{Diving into the planetary system of Proxima with NIRPS: Breaking the metre per second barrier in the infrared}. 2025, \href{https://doi.org/10.1051/0004-6361/202553728}{A\&A, 700, A11}.}
\cvitem{}{Doyon, R. et al. (\langen{including}\langfr{incluant} \textbf{Vandal, T.}) \textit{NIRPS Joins HARPS: Setting New Standards at Infrared Wavelengths}. 2025, \href{https://doi.org/10.18727/0722-6691/5379}{The Messenger, 194, 13}.}
\cvitem{}{Deslières, A. et al. (\langen{including}\langfr{incluant} \textbf{Vandal, T.}) \textit{The Gl 229 System Revisited with the Line-by-line Framework: Planetary Signals Now Appear as Stellar Activity Ghosts}. 2025, \href{https://doi.org/10.3847/1538-3881/ada77a}{AJ, 169, 182}.}
\cvitem{}{Albert, L. et al. (\langen{including}\langfr{incluant} \textbf{Vandal, T.}) \textit{JWST 1.5 {\ensuremath{\mu}}m and 4.8 {\ensuremath{\mu}}m Photometry of Y Dwarfs}. 2025, \href{https://doi.org/10.3847/1538-3881/adadf9}{AJ, 169, 163}.}
\cvitem{}{Blakely, D. et al. (\langen{including}\langfr{incluant} \textbf{Vandal, T.}) \textit{The James Webb Interferometer: Space-based Interferometric Detections of PDS 70 b and c at 4.8 {\ensuremath{\mu}}m}. 2025, \href{https://doi.org/10.3847/1538-3881/ad9b94}{AJ, 169, 137}.}
\cvitem{}{Cooper, R. et al. (\langen{including}\langfr{incluant} \textbf{Vandal, T.}) \textit{Commissioning and calibration of the JWST aperture masking interferometry mode}. 2024, \href{https://doi.org/10.1117/12.3019153}{Optical and Infrared Interferometry and Imaging IX, 13095, 130952R}.}
\cvitem{}{Malo, L. et al. (\langen{including}\langfr{incluant} \textbf{Vandal, T.}) \textit{NIRPS near-infrared spectrograph: AITV phase at ESO3.6m/La Silla}. 2024, \href{https://doi.org/10.1117/12.3020168}{Ground-based and Airborne Instrumentation for Astronomy X, 13096, 1309646}.}
\cvitem{}{Artigau, É. et al. (\langen{including}\langfr{incluant} \textbf{Vandal, T.}) \textit{NIRPS first light and early science: breaking the 1 m/s RV precision barrier at infrared wavelengths}. 2024, \href{https://doi.org/10.1117/12.3018994}{Ground-based and Airborne Instrumentation for Astronomy X, 13096, 130960C}.}
\cvitem{}{Jahandar, F. et al. (\langen{including}\langfr{incluant} \textbf{Vandal, T.}) \textit{Comprehensive High-resolution Chemical Spectroscopy of Barnard's Star with SPIRou}. 2024, \href{https://doi.org/10.3847/1538-4357/ad3063}{The Astrophysical Journal, 966, 56}.}
\cvitem{}{Moutou, C. et al. (\langen{including}\langfr{incluant} \textbf{Vandal, T.}) \textit{Characterizing planetary systems with SPIRou: M-dwarf planet-search survey and the multiplanet systems GJ 876 and GJ 1148}. 2023, \href{https://doi.org/10.1051/0004-6361/202346813}{A\&A, 678, A207}.}
\cvitem{}{Doyon, R. et al. (\langen{including}\langfr{incluant} \textbf{Vandal, T.}) \textit{The Near Infrared Imager and Slitless Spectrograph for the James Webb Space Telescope. I. Instrument Overview and In-flight Performance}. 2023, \href{https://doi.org/10.1088/1538-3873/acd41b}{PASP, 135, 098001}.}
\cvitem{}{Cortés-Zuleta, P. et al. (\langen{including}\langfr{incluant} \textbf{Vandal, T.}) \textit{Optical and near-infrared stellar activity characterization of the early M dwarf Gl 205 with SOPHIE and SPIRou}. 2023, \href{https://doi.org/10.1051/0004-6361/202245131}{A\&A, 673, A14}.}
\cvitem{}{Calissendorff, P. et al. (\langen{including}\langfr{incluant} \textbf{Vandal, T.}) \textit{JWST/NIRCam Discovery of the First Y+Y Brown Dwarf Binary: WISE J033605.05-014350.4}. 2023, \href{https://doi.org/10.3847/2041-8213/acc86d}{The Astrophysical Journal, 947, L30}.}
\cvitem{}{Rigby, J. et al. (\langen{including}\langfr{incluant} \textbf{Vandal, T.}) \textit{The Science Performance of JWST as Characterized in Commissioning}. 2023, \href{https://doi.org/10.1088/1538-3873/acb293}{PASP, 135, 048001}.}
\cvitem{}{Kiefer, F. et al. (\langen{including}\langfr{incluant} \textbf{Vandal, T.}) \textit{A sub-Neptune planet around TOI-1695 discovered and characterized with SPIRou and TESS}. 2023, \href{https://doi.org/10.1051/0004-6361/202245129}{A\&A, 670, A136}.}
\cvitem{}{Sivaramakrishnan, A. et al. (\langen{including}\langfr{incluant} \textbf{Vandal, T.}) \textit{The Near Infrared Imager and Slitless Spectrograph for the James Webb Space Telescope. IV. Aperture Masking Interferometry}. 2023, \href{https://doi.org/10.1088/1538-3873/acaebd}{PASP, 135, 015003}.}
\cvitem{}{Kammerer, J. et al. (\langen{including}\langfr{incluant} \textbf{Vandal, T.}) \textit{The Near Infrared Imager and Slitless Spectrograph for JWST. V. Kernel Phase Imaging and Data Analysis}. 2023, \href{https://doi.org/10.1088/1538-3873/ac9a74}{PASP, 135, 014502}.}
\cvitem{}{Cook, N. et al. (\langen{including}\langfr{incluant} \textbf{Vandal, T.}) \textit{APERO: A PipelinE to Reduce Observations-Demonstration with SPIRou}. 2022, \href{https://doi.org/10.1088/1538-3873/ac9e74}{PASP, 134, 114509}.}
\cvitem{}{Cadieux, C. et al. (\langen{including}\langfr{incluant} \textbf{Vandal, T.}) \textit{TOI-1452 b: SPIRou and TESS Reveal a Super-Earth in a Temperate Orbit Transiting an M4 Dwarf}. 2022, \href{https://doi.org/10.3847/1538-3881/ac7cea}{AJ, 164, 96}.}
\cvitem{}{Artigau, É. et al. (\langen{including}\langfr{incluant} \textbf{Vandal, T.}) \textit{Line-by-line Velocity Measurements: an Outlier-resistant Method for Precision Velocimetry}. 2022, \href{https://doi.org/10.3847/1538-3881/ac7ce6}{AJ, 164, 84}.}
\cvitem{}{Kammerer, J. et al. (\langen{including}\langfr{incluant} \textbf{Vandal, T.}) \textit{Performance of near-infrared high-contrast imaging methods with JWST from commissioning}. 2022, \href{https://doi.org/10.1117/12.2628865}{Space Telescopes and Instrumentation 2022: Optical, Infrared, and Millimeter Wave, 12180, 121803N}.}
\cvitem{}{Martioli, E. et al. (\langen{including}\langfr{incluant} \textbf{Vandal, T.}) \textit{TOI-1759 b: A transiting sub-Neptune around a low mass star characterized with SPIRou and TESS}. 2022, \href{https://doi.org/10.1051/0004-6361/202142540}{A\&A, 660, A86}.}
\cvitem{}{Artigau, É. et al. (\langen{including}\langfr{incluant} \textbf{Vandal, T.}) \textit{TOI-1278 B: SPIRou Unveils a Rare Brown Dwarf Companion in Close-in Orbit around an M Dwarf}. 2021, \href{https://doi.org/10.3847/1538-3881/ac096d}{AJ, 162, 144}.}
\cvitem{}{Pelletier, S. et al. (\langen{including}\langfr{incluant} \textbf{Vandal, T.}) \textit{Where Is the Water? Jupiter-like C/H Ratio but Strong H<SUB>2</SUB>O Depletion Found on {\ensuremath{\tau}} Bo{\"o}tis b Using SPIRou}. 2021, \href{https://doi.org/10.3847/1538-3881/ac0428}{AJ, 162, 73}.}

% BUG: Update with latest talks (ExSS, stsci, ESO Vitacura seminar, SF2A)
% TODO: Restructure this section. Use tak title as the main item and put events/meetings as secondary field. Could also make enumeartions more dense when same talk was presented multiple times.
\subsection{\langen{Oral Presentations (Conferences and Seminars)}\langfr{Présentations orales (conférences et séminaires)}}
\cventry{2024}{From HR 8799 to Y-dwarf binaries: Understanding planet formation across the stellar IMF}{Extreme Solar Systems V}{Christchurch}{}{\normalsize \textit{with JWST Interferometry} }
\cventry{2023}{A Kernel Phase Pipeline for High-Contrast Imaging below the Diffraction Limit with JWST}{Improving JWST Data Products Workshop}{Baltimore}{}{}
\cventry{2022}{Infrared interferometric imaging below the diffraction limit with JWST}{\langen{Center for research in astrophysics of Quebec (CRAQ) annual meeting}\langfr{Réunion annuelle du Centre de recherche en astrophysique du Québec (CRAQ)}}{Orford}{}{}
\cventry{2022}{\irex}{\langen{IREx Café}\langfr{Café IREx}}{Montréal}{\langen{Seminar}\langfr{Séminaire}}{2022: \textit{Introduction to Aperture Masking and Kernel Phase Interferometry}\\2021: \textit{Introduction to Hamiltonian Monte-Carlo}\\2019: \textit{Gaussian Processes and their Applications in Astrophysics}}
\cventry{2021}{Long term RV trend analysis and correction}{\langen{SPIRou Legacy Survey Science Meeting}\langfr{Rencontre de l'équipe scientifique SPIRou}}{Montréal}{}{}
\cventry{2019}{Disentangling Stellar Activity in $\beta$ Pictoris with Gaussian Processes}{\langen{Canadian Undergraduate Physics Conference (CUPC)}\langfr{Conférence canadienne des étudiants en physique (CCEP/CUPC)}}{Montréal}{}{}

% BUG: Update with latest posters (CRAQ, Cool stars)
% TODO: Restructure this section. Use title as the main item and put events/meetings as secondary field. Could also make enumerations more dense when same poster was presented multiple times.
\subsection{\langen{Poster Presentations}\langfr{Présentations avec affiches}}
\cventry{2022}{Detecting Hot Close-in Gas Giants Through Infrared High-Dispersion Spectroscopy}{\langen{Canadian Astronomical Society (CASCA) Annual Meeting}\langfr{Rencontre annuelle de la société canadienne d'astronomie (CASCA)}}{\langen{Online}\langfr{En ligne}}{}{\langen{Available online}\langfr{Disponible en ligne}: \url{https://vandalt.github.io/casca2022/}}
\cventry{2021}{Chasseurs d'exoplanètes: à la recherche de planètes habitables avec les élèves du secondaire}{ComSciCon-QC}{\langen{Online}\langfr{En ligne}}{}{\langen{Available online}\langfr{Disponible en ligne}: \url{https://vandalt.github.io/poster-comscicon-sprint-k218/}}
% HACK: \newline{} does nothing and \\ does not work in description field so splitting and enlRging
\cventry{2020}{Dynamical Mass Estimates of the beta Pictoris Planetary System through Gaussian Process}{Exoplanets III}{\langen{Online}\langfr{En ligne}}{}{\normalsize \textit{Stellar Activity Modeling}}
\cventry{2020}{Dynamical Mass Estimates of the beta Pictoris Planetary System through Gaussian Process}{\langen{Canadian Astronomical Society (CASCA) Annual Meeting}\langfr{Rencontre annuelle de la société canadienne d'astronomie (CASCA)}}{\langen{Online}\langfr{En ligne}}{}{\normalsize \textit{Stellar Activity Modeling}}

% BUG: Add refereeing in separate section.
\section{\langfr{Enseignement et tutorat}\langen{Teaching and Mentoring}}
\cventry{2020-...}{Université de Montréal}{\langen{Teaching Assistant}\langfr{Auxiliaire d'enseignement}}{Montréal}{}{
  PHY3051/6051 - \langen{Modern data analysis in physics}\langfr{Analyse moderne des données physiques} (\langen{Joint class}\langfr{Cours conjoint} BSc/PhD, \langen{Winter}\langfr{Hiver} 2022-2024) \\
  PHY1234 - \langen{Introduction to numerical physics}\langfr{Introduction à la physique numérique} (\langen{Fall}\langfr{Automne} 2021, 2023) \\
  PHY1901 - \langen{Classical mechanics and modern physics}\langfr{Mécanique et physique moderne} (\langen{Fall}\langfr{Automne} 2020)
}
\cventry{2024}{Centre de services scolaire de Montréal}{\langen{Inidividual Mentoring}\langfr{Mentorat individual}}{Montréal}{}{\langen{One-to-one meetings with a gifted elementary school student}\langfr{Rencontres de mentorat avec un élève doué du primaire}}
\cventry{2016-2017}{Cégep de Shawinigan}{\langen{Individual Tutoring}\langfr{Tutorat individuel}}{Shawinigan}{}{\langen{Physics, mathematics, French literature}\langfr{Physique, mathématiques, littérature}}

\section{\langen{Additional scientific, volunteering and community activities}\langfr{Implication scientifique, bénévole et sociale}}
% TODO: conference ?attendance? do people put this on their CV?
\subsection{\langfr{Participation à des conférences}\langen{Conferences and workshop}}
% BUG: Add 2023-2024 (dotastro, exss, stsci, coolstars, CASCA)
% TODO: SPIE acronym
\cventry{2023}{Space Telescope Science Institute}{Improving JWST Data Products Workshop}{Baltimore}{}{}
\cventry{2023}{Flatiron Institute}{Dot Astronomy 12}{New-York}{}{\langen{Workshop in an "unconference" format about the use of modern computing facilities and software in astronomy}\langfr{Atelier sous format \guillemotleft{}~unconference~\guillemotright{} sur les outils numériques en astronomie}}
\cventry{2022}{SPIE}{SPIE Astronomical Telescopes and Instrumentation}{Montréal}{}{}
\cventry{2021}{University of California Irvine}{Aperture Masking and Kernel Phase Hackathon}{\langen{Online}\langfr{En ligne}}{}{\langen{Presentations and discussions about state-of-the-art techniques and standards for the years to come}\langfr{Présentations et ateliers pour discuter des méthodes de pointe et des standards pour les prochaines années.}}
\cventry{2021}{\langen{Center for research in astrophysics of Quebec (CRAQ)}\langfr{Centre de recherche en astrophysique du Québec (CRAQ)}}{AstroComm}{\langen{Online}\langfr{En ligne}}{}{\langen{Workshop about science communication for graduate students in astronomy}\langfr{Ateliers de communication scientifique pour les étudiant$\cdot$e$\cdot$s aux cycles supérieurs en astronomie.}}
\cventry{2021}{\langen{ComSciCon workshop series}\langfr{Série de conférences ComSciCon}}{ComSciCon-QC}{\langen{Online}\langfr{En ligne}}{}{\langen{Workshop about science communication for graduate students}\langfr{Conférence sur la communication scientifique pour les étudiant$\cdot$e$\cdot$s aux cycles supérieurs.}}
\cventry{2020-2022}{\langen{Canadian Astronomical Society (CASCA)}\langfr{Société canadienne d'astronomie (CASCA)}}{\langen{CASCA Annual Meeting}\langfr{Rencontre annuelle de la CASCA}}{\langfr{En ligne}\langen{Online}}{}{}
\cventry{2019}{McGill University}{\langen{Canadian Undergraduate Physics Conference (CUPC)}\langfr{Conférence canadienne des étudiants en physique (CCEP/CUPC)}}{Montréal}{}{}

% TODO: Keep an excel file with history of this
% TODO: Update counts
% TODO: Add eclipse-related things
% TODO: Cegep shawi
% TODO: Condenser certaines implications genre bénévolat dans événements et présentations?
\subsection{\langfr{Communication scientifique et science citoyenne}\langen{Science Communication and Citizen Science}}
\cventry{2022-2025}{\irex}{\langfr{Journées carrières de l'IREx}\langen{IREx Career Days}}{Montréal}{\langen{Volunteer}\langfr{Bénévole}}{\langen{Presentation and discussion with high-school students}\langfr{Présentation et discussion avec des jeunes du secondaires et du cégep visitant l'institut.}}
% TODO: 2023, 2024? Not sure
\cventry{2021 \langen{and}\langfr{et} 2022}{\irex}{\langen{Summer Interns Welcome Day}\langfr{Journée d'accueil des stagiaires d'été}}{Montréal}{\langen{Presenter}\langfr{Présentateur}}{\langen{Presentation about tools used in astronomy}\langfr{Présentation des outils utilisés en astronomie}}
% TODO: Other years?
\cventry{2021}{\irex}{Grande conférence de l'IREx 2021}{Montréal}{\langfr{Bénévole}\langen{Volunteer}}{\langfr{Accueil du public et participation au segment questions-réponses à la fin de la conférence}\langen{Welcoming the public and taking part in the Q\&A at the end}}
\cventry{2021-2022}{InitiaSciences}{InitiaSciences}{Montréal}{\langen{Treasurer and founding member}\langfr{Trésorier et membre fondateur}}{\langen{Non-profit enabling high-school students to do research, supervisted by graduate students}\langfr{Organisme permettant aux jeunes du secondaire et du cégep d'effectuer de la recherche en milieu universitaire}}
\cventry{2021-...}{\irex \langen{and}\langfr{et} 24h de sciences}{\langfr{Présentations }\textit{Un astronome dans votre classe}\langen{ Conferences}}{Montréal}{\langen{Speaker}\langfr{Conférencier}}{\langen{Conference about astronomy for elementary schools}\langfr{Présentation sur l'astronomie dans les écoles primaires}, 3 \langen{presentations}\langfr{présentations}}
% HACK: Moved conférencier to put after RCM
% TODO: Format better (shorter)
\cventry{2021-...}{\irex \langen{and}\langfr{et}}{\langfr{Conférences }\textit{Ma vie de chercheur en astronomie}\langen{ Conferences}}{Montréal}{}{{\normalsize \textit{Regroupement des cégeps de Montréal (programme acceScience), \langfr{Conférencier}\langen{Speaker}}}\newline 5 \langfr{présentations}\langen{presentations}}
\cventry{2021}{Projet SEUR, Université de Montréal}{Séjours d'immersion}{Montréal}{\langen{Speaker}\langfr{Conférencier}}{\langen{Presentation about studying physics and exoplanet research}\langfr{Présentation sur les études en physique et atelier sur les exoplanètes}}
\cventry{2020-2025}{Coeur des sciences, Université du Québec à Montréal}{Sprint de sciences}{Montréal}{\langen{Speaker}\langfr{Conférencier}}{\langen{Interactive conference about exoplanets for high-schools}\langfr{Atelier interactif sur les exoplanètes au secondaire}, \langen{approximately}\langfr{environ} 10 \langen{per semester}\langfr{par session}}
\cventry{2020-...}{\langen{Contribution to software projects used in astronomy}\langfr{Contribution à plusieurs logiciels utilisés en astronomie}}{\langen{Open-source software contributions}\langfr{Contributions à des logiciels libres}}{}{}{\langen{Full list available on}\langfr{Liste complète disponible sur} GitHub: \url{https://github.com/vandalt}}
\cventry{2020 \langen{and}\langfr{et} 2021}{\langen{Center for research in astrophysics of Quebec (CRAQ)}\langfr{Centre de recherche en astrophysique du Québec (CRAQ)}}{\langen{CRAQ Calendar}\langfr{Calendrier du CRAQ}}{Montréal}{\langen{Volunteer}\langfr{Bénévole}}{\langen{Translation and management of the distribution list}\langfr{Traduction de textes et mise à jour de la liste de distribution}}
% TODO: Was there one after that? Eclipse?
\cventry{2018 \langen{and}\langfr{et} 2019}{Université de Montréal}{AstroMIL: \langfr{Journée d'astronomie du Campus MIL}\langen{Astronomy Day on the MIL Campus}}{Montréal}{\langen{Volunteer}\langfr{Bénévole}}{}
\cventry{2018 \langen{and}\langfr{et} 2022}{CRAQ \langen{and}\langfr{et} Faculté des Arts et Sciences de l'Unviersité de Montréal}{\langfr{Festival }Eurêka!\langen{ Festival}}{Montréal}{\langen{Presenter}\langfr{Animateur}}{}
% TODO: Remove?
%\cventry{2015}{Réseau technoscience}{Hydro-Québec Science Fair}{}{}{
%\begin{itemize}
%    \item Hydro-Québec Science Fair Maurcie-Centre-du-Québec regional finals, Trois-Rivières
%    \item Hydro-Québec Science Fair Québec Final, Gatineau
%\end{itemize}
%}
\subsection{Organisation d'événements}
\cventry{2019}{\langen{McGill University}\langfr{Université McGill}}{\langen{Canadian Undergraduate Physics Conference (CUPC)}\langfr{Conférence canadienne des étudiants en physique (CCEP/CUPC)}}{Montréal}{\langen{Vice-president of events}\langfr{Vice-président aux événements}}{\langen{Sponsorship search and event planification}\langfr{Recherche de commandites et planification des événements}}
%\subsection{Refereeing (soccer)}
%\cventry{2018 and 2019}{Soccer Québec}{Nutrilait International U14}{Montréal}{}{Assigned as an assistant referee for the 2019 final game}
%\cventry{2018}{US Soccer Academy Program}{Needham Memorial Day Tournament}{Amherst, MA}{représenting Soccer Québec}{Named best assistant referee of the tournament}
%\cventry{2015 and 2017}{Soccer Québec}{Tournoi des sélections régionales}{}{}{Soccer tournament similar to Jeux du Québec, every two years \\ Assigned as an assistant referee for the 2017 final game}
%\subsection{Student sport}
%\cventry{2015 and 2016}{RSEQ}{Collège Shawinigan}{Shawinigan}{}{Cross-country (Provincial championship, 2015 and 2016)
%}
%\cventry{2010-2015}{RSEQ}{École Secondaire du Rocher}{Shawinigan}{}{
%Basket-ball (2013 and 2015) and cross-country (2011-2014) provincial championships
%}

\section{\langen{Awards \& Recognitions}\langfr{Prix, bourses et mentions}}
\subsection{Université de Montréal}
\cventry{2022}{Études supérieures et postdoctorales (ESP), Université de Montréal}{\langfr{Bourse d'excellence des ESP - Bourse Alma Mater}\langen{ESP Excellence Scholarship - Alma Mater}}{}{\money{10000}}{}
\cventry{2022}{\irex}{\langen{Lumbroso IREx Ambassador Grant}\langfr{Bourse Lumbroso pour ambassadeur de l'IREx}}{}{\money{2000}}{}
\cventry{2022-2026}{Fonds de recherche du Québec - Nature et technologies}{\langfr{Bourse de doctorat en recherche (B2X)}\langen{Doctoral research scholarship (B2X)}}{}{\money{84000}}{\langen{\money{21000} per year for 4 years}\langfr{\money{21000} par an pour un maximum de 4 ans}}
\cventry{2022}{\langen{Department of Physics}\langfr{Département de physique}, Université de Montréal}{\langen{Teaching Award}\langfr{Prix d'enseignement} \langen{"Petit Nobel - NanoNobel"}\langfr{\guillemotleft{} Petit Nobel - NanoNobel \guillemotright{}}}{}{}{\langen{Excellence in teaching award for a teaching assistant, voted by 3rd year undergraduate students}\langfr{Prix d'excellence en enseignement décerné à un auxiliaire par un vote des étudiant$\cdot$es de 3e année}}
\cventry{2022}{Université de Montréal}{\langen{Financial support scholarship}\langfr{Bourse de soutien financier}}{}{\money{3000}}{\langen{Awarded for the excellence of my 2021-2022 NSERC application}\langfr{Pour l'excellence du dossier de candidature au concours 2021-2022 de la bourse du CRSNG}}
\cventry{2022}{Fondation des Anciens de Shawinigan}{Bourse d'études 3e cycle\langen{ (Doctoral Scholarship)}}{}{\money{5000}}{}
\cventry{2021}{Université de Montréal}{Bourse pour passage accéléré au doctorat\langen{ (Scholarship for fast track to PhD)}}{}{\money{7000}}{}
\cventry{2021}{Fonds de recherche du Québec - Nature et technologies}{\langen{Master's research scholarship (B1X)}\langfr{Bourse de maîtrise en recherche (B1X)}}{}{\money{21000}}{\langen{Value adjusted for PhD after fast track}\langfr{Montant ajusté pour le doctorat après le passage accéléré}}
\cventry{2020}{Fondation des Anciens de Shawinigan}{Bourse d'études 2e cycle\langen{ (Master's Scholarship)}}{}{\money{2000}}{}
\cventry{2020}{\langen{NSERC (CREATE program)}\langfr{CRSNG (programme CREATE)}}{\langfr{Bourse \textit{Technologies for Exo Planetary Science}}\langen{Technologies for Exo Planetary Science Scholarship}}{}{\money{7500}}{}
\subsection{McGill University}
\cventry{2018}{\irex}{\langen{Trottier Excellence Grant for Summer Interns}\langfr{Bourse d'excellence Trottier pour stagiaires d'été}}{}{\money{6000}}{}
\cventry{2017}{\langen{Entrance Scholarship, McGill University}\langfr{Bourse d'entrée au baccalauréat, Univeristé McGill}}{R.E. Powell Scholarship}{}{\money{5000}}{}
\subsection{Cégep de Shawinigan}
\cventry{2017}{\langen{Highest R-score, Cégep de Shawinigan}\langfr{Plus haute cote de rendement, Cégep de Shawinigan}}{\langen{Governor General's Academic Medal}\langfr{Médaille académique du gouverneur général}}{}{}{}
%\cventry{2017}{For academic excellence in grades}{Athéna award in the Natural Sciences program}{}{}{}
%\cventry{2017}{For academic excellence in grades}{Athéna award for pre-university general education courses}{}{}{}
%\cventry{2017}{For understanding of the subject, relevance and présentation quality}{1st place - Applied sciences, Science Symposium}{}{}{Final project "Lumière sur les trous noirs"}
%\cventry{2017}{Awarded by the audience (high-school students and general public)}{Prix Coup de coeur, Science Symposium}{}{}{Final project "Lumière sur les trous noirs"}
%\cventry{2016 and 2017}{Student best conciliating sport and studies}{Bourse de la performance académique et sportive}{}{}{}
%\cventry{2016}{For academic excellence in grades (CEGEP)}{Desjardins Scholarship}{}{}{}
%\cventry{2016}{1st place - Collège Shawinigan}{American Mathematics Competitions}{}{}{}
%\subsection{École secondaire du Rocher}
%\cventry{2015}{Highest average}{Governor General's Academic Medal}{}{}{}
%\cventry{2015}{Male Student-athlete best conciliating sport and studies in Quebec}{Sablon d'or}{}{}{Provincial RSEQ Gala, Sherbrooke}
%\cventry{2015}{Winning project, all categories}{Hydro-Québec first prize}{}{}{Regional Hydro-Québec Science Fair, Mauricie et Centre-du-Québec}
%\cventry{2015}{For receiving "Student of the year" award every year in secondary school}{ESDR Award}{}{}{}


\end{document}
